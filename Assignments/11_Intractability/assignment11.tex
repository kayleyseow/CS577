% with answers
\documentclass[solutionorbox,answers]{exam}
% without answers
% \documentclass[solutionorbox]{exam}

%%%%%%%%%%%%%%%%%%%%%%%%%%%%%%%%%%%%%%%%%%%%%%%%%%%%%%%%%%%%%%%
% Update to change header
\newcommand{\courseName}{CS 577}
\newcommand{\assignmentName}{Assignment 11 -- Intractability}
\newcommand{\semester}{Spring 2023}
%%%%%%%%%%%%%%%%%%%%%%%%%%%%%%%%%%%%%%%%%%%%%%%%%%%%%%%%%%%%%%%

\usepackage[utf8]{inputenc}
\usepackage[T1]{fontenc}

\usepackage{amsmath}
\usepackage{amsfonts}
\usepackage{amsthm}
\usepackage[mathscr]{eucal}
\usepackage{booktabs}
\usepackage{tkz-graph}
\usepackage[ruled]{algorithm2e}
\usepackage{graphicx}
\usepackage{enumitem}

\usepackage{hyperref}

\pagestyle{headandfoot}
\runningheadrule
\firstpageheader{\courseName}{\huge \assignmentName}{\semester}
\runningheader{\courseName}
{\assignmentName}
{\semester}
\firstpagefooter{}{}{}
\runningfooter{}{Page \thepage\ of \numpages}{}

\begin{document}

\begin{center}
\fbox{\parbox{5.5in}{\centering
Answer the questions in the boxes provided on the
question sheets. If you run out of room for an answer,
add a page to the end of the document. \\
\vspace{0.1in}
}}
\end{center}
\vspace{0.1in}
\makebox[0.48\textwidth]{Name:\enspace\hrulefill} \qquad
\makebox[0.48\textwidth]{Wisc id:\enspace\hrulefill}

\begin{questions}

\section*{Intractibility}

\question  \textit{Kleinberg, Jon. Algorithm Design (p. 506, q. 4).} A system has a set of $n$ processes and a set of $m$ resources. 
At any point in time, each process specifies a set of resources that it requests to use. 
Each resource might be requested by many processes at once; but it can only be used by a single process at a time. 
If a process is allocated all the resources it requests, then it is active; otherwise it is blocked. 

Thus we phrase the Resource Reservation Problem as follows: 
Given a set of processes and resources, the set of requested resources for each process, and a number $k$, is it possible to allocate resources to processes so that at least $k$ processes will be active?

For the following problems, either give a polynomial-time algorithm or prove the problem is NP-complete.

\begin{parts}
\part The general Resource Reservation Problem defined above.
  \begin{solutionbox}{\stretch 1} \vspace{1em} 
  \end{solutionbox}


\newpage

\part The special case of the problem when $k = 2$.
  \begin{solutionbox}{\stretch 1} \vspace{1em} \\
  \end{solutionbox}


\part The special case of the problem when there are two types of resources--say, people and equipment--and each process requires at most one resource of each type (In other words, each process requires one specific person and one specific piece of equipment.)
  \begin{solutionbox}{\stretch 1} \vspace{1em} 
  \end{solutionbox}

\part The special case of the problem when each resource is requested by at most two processes.


  \begin{solutionbox}{\stretch 1} \vspace{1em} \\
  \end{solutionbox}

\end{parts}

\newpage

\question  \textit{Kleinberg, Jon. Algorithm Design (p. 506, q. 7).} The 3-Dimensional Matching Problem is an NP-complete problem defined as follows:

Given disjoint sets $X$, $Y$, and $Z$, each of size $n$, and given a set $T\subseteq X\times Y\times Z$ of ordered triples, does there exist a set of $n$ triples in T that each element of $X\cup Y\cup Z$ is contained in exactly one of these triples?

Since 3-Dimensional Matching is NP-complete, it is natural to expect that the 4-Dimensional Problem is at least as hard. 

Let us define 4-Dimensional Matching as follows. Given sets $W$, $X$, $Y$, and $Z$, each of size $n$, and a collection $C$ of ordered 4-tuples of the form $(w_i, x_j, y_k, z_\ell)$, do there exist $n$ 4-tuples from $C$ such that each element of $W \cup Y \cup X \cup Z$ appears in exactly one of these
4-tuples?

Prove that 4-Dimensional Matching is NP-complete. Hint: use a reduction from 3-Dimensional Matching.

  \begin{solutionbox}{\stretch 1} \vspace{1em} 
  \end{solutionbox}

\newpage


  \question \textit{Kleinberg, Jon. Algorithm Design (p. 507, q. 6).} Consider an instance of the Satisfiability Problem, specified by clauses $C_1,...,C_m$ over a set of Boolean variables $x_1,...,x_n$. We say that the instance is monotone if each term in each clause consists of a nonnegated variable; that is, each term is equal to $x_i$, for some $i$, rather than $\overline{x_i}$. Monotone instances of Satisfiability are very easy to solve: They are always satisfiable, by setting each variable equal to 1.

For example, suppose we have the three clauses
\begin{center}$(x_1\lor x_2),(x_1\lor x_3),(x_2\lor x_3)$.\end{center}

This is monotone, and the assignment that sets all three variables to 1 satisfies all the clauses. But we can observe that this is not the only satisfying assignment; we could also have set $x_1$ and $x_2$ to 1, and $x_3$ to 0. Indeed, for any monotone instance, it is natural to ask how few variables we need to set to 1 in order to satisfy it.

Given a monotone instance of Satisfiability, together with a number $k$, the problem of \textit{Monotone Satisfiability with Few True Variables} asks: Is there a satisfying assignment for the instance in which at most $k$ variables are set to 1? Prove this problem is NP-complete.



  \begin{solutionbox}{\stretch 1} \vspace{1em} 
  \end{solutionbox}

\newpage

\question \textit{Kleinberg, Jon. Algorithm Design (p. 509, q. 10).} Your friends at WebExodus have recently been doing some consulting work for companies that maintain large, publicly accessible Web sites and they’ve come across the following Strategic Advertising Problem.

A company comes to them with the map of a Web site, which we’ll model as a directed graph $G = (V, E)$. The company also provides a set of $t$ trails typically followed by users of the site; we’ll model these trails as directed paths $P_1, P_2,..., P_t$ in the graph $G$ (i.e., each $P_i$ is a path in $G$).

The company wants WebExodus to answer the following question for them: 
Given $G$, the paths $\{P_i\}$, and a number $k$, is it possible to place
advertisements on at most $k$ of the nodes in $G$, so that each path $P_i$
includes at least one node containing an advertisement? We’ll call this
the Strategic Advertising Problem, with input $G, \{P_i : i = 1,...,t\}$, and $k$.
Your friends figure that a good algorithm for this will make them all
rich; unfortunately, things are never quite this simple.

\begin{parts}

\part Prove that Strategic Advertising is NP-Complete.

  \begin{solutionbox}{\stretch 1} \vspace{1em} 
  \end{solutionbox}
\newpage

\part Your friends at WebExodus forge ahead and write a pretty fast algorithm $\mathscr{S}$ that produces yes/no answers to arbitrary instances of the Strategic Advertising Problem. You may assume that the algorithm $\mathscr{S}$ is always correct.

Using the algorithm $\mathscr{S}$ as a black box, design an algorithm that takes input $G, \{P_i : i = 1,...,t\}$, and $k$ as in part (a), and does one of the following two things:

\begin{itemize}
\item Outputs a set of at most $k$ nodes in $G$ so that each path $P_i$ includes at least one of these nodes.
\item Outputs (correctly) that no such set of at most $k$ nodes exists.
\end{itemize}

Your algorithm should use at most polynomial number of steps, together with at most polynomial number of calls to the algorithm $\mathscr{S}$.

  \begin{solutionbox}{\stretch 1} \vspace{1em} \\
  \end{solutionbox}

\end{parts}

\end{questions}

\end{document}




