% with answers
\documentclass[solutionorbox,answers]{exam}
% without answers
% \documentclass[solutionorbox]{exam}

%%%%%%%%%%%%%%%%%%%%%%%%%%%%%%%%%%%%%%%%%%%%%%%%%%%%%%%%%%%%%%%
% Update to change header
\newcommand{\courseName}{CS 577}
\newcommand{\assignmentName}{Assignment 13 -- Randomization}
\newcommand{\semester}{Spring 2023}
%%%%%%%%%%%%%%%%%%%%%%%%%%%%%%%%%%%%%%%%%%%%%%%%%%%%%%%%%%%%%%%

\usepackage[utf8]{inputenc}
\usepackage[T1]{fontenc}

\usepackage{amsmath}
\usepackage{amsfonts}
\usepackage{amsthm}
\usepackage[mathscr]{eucal}
\usepackage{booktabs}
\usepackage{tkz-graph}
\usepackage[ruled]{algorithm2e}
\usepackage{graphicx}
\usepackage{enumitem}

\usepackage{hyperref}

\pagestyle{headandfoot}
\runningheadrule
\firstpageheader{\courseName}{\huge \assignmentName}{\semester}
\runningheader{\courseName}
{\assignmentName}
{\semester}
\firstpagefooter{}{}{}
\runningfooter{}{Page \thepage\ of \numpages}{}

\begin{document}

\begin{center}
\fbox{\parbox{5.5in}{\centering
Answer the questions in the boxes provided on the
question sheets. If you run out of room for an answer,
add a page to the end of the document. \\
\vspace{0.1in}
}}
\end{center}
\vspace{0.1in}
\makebox[0.48\textwidth]{Name:\enspace\hrulefill} \qquad
\makebox[0.48\textwidth]{Wisc id:\enspace\hrulefill}

\begin{questions}

\section*{Randomization}

\question  \textit{Kleinberg, Jon. Algorithm Design (p. 782, q. 1).} 

\textit{3-Coloring} is a yes/no question, but we can phrase it as an optimization problem as follows.

Suppose we are given a graph $G=(V,E)$, and we want to color each node with one of three colors,
even if we aren't necessarily able to give different colors to every pair of adjacent nodes. Rather, we
say that an edge $(u,v)$ is \textit{satisfied} if the colors assigned to $u$ and $v$ are different. Consider a $3$-coloring
that maximizes the number of satisfied edges, and let $c^*$ denote this number. Give a polynomial-time
algorithm that produces a $3$-coloring that satisfies at least $\frac{2}{3}c^*$ edges. If you want, your algorithm can
be randomized; in this case, the expected number of edges it satisfies should be at least $\frac{2}{3}c^*$.

\begin{solutionbox}{\stretch 1} \vspace{1em}

 % begin your solution here %

\end{solutionbox}

\newpage

\question  \textit{Kleinberg, Jon. Algorithm Design (p. 787, q. 7).}

In lecture, we designed an approximation algorithm to within a factor of $7/8$ for the MAX $3$-SAT
Problem, where we assumed that each clause has terms associated with three different variables. In this
problem, we will consider the analogous MAX SAT Problem: Given a set of clauses $C_1, \cdots, C_k$ over a
set of variables $X=\{x_1,\cdots, x_n\}$, find a truth assignment satisfying as many of the clauses as possible.
Each clause has at least one term in it, and all the variables in a single clause are distinct, but otherwise
we do not make any assumptions on the length of the clauses: There may be clauses that have a lot of
variables, and others may have just a single variable.

\begin{parts}
\part 
First consider the randomized approximation algorithm we used for MAX $3$-SAT, setting each
variable independently to true or false with probability $1/2$ each. Show that in the MAX SAT, the expected number
of clauses satisfied by this random assignment is at least $k/2$, that is, at least half of the clauses are
satisfied in expectation.

  \begin{solutionbox}{\stretch 1} \vspace{1em} 

     % begin your solution here %

  \end{solutionbox}

\part Give an example to show that there are MAX SAT instances such that no
assignment satisfies more than half of the clauses.

\begin{solutionbox}{\stretch 1} \vspace{1em} 

  % begin your solution here %

\end{solutionbox}

\newpage
\part If we have a clause that consists only of a single term (e.g., a clause consisting just of $x_1$, or just
of $\overline{x_2}$), then there is only a single way to satisfy it: We need to set the corresponding variable in
the appropriate way. If we have two clauses such that one consists of just the term $x_i$, and the
other consists of just the negated term $\overline{x_i}$, then this is a pretty direct contradiction. Assume that
our instance has no such pair of "conflicting clauses"; that is, for no variable $x_i$ do we have both a
clause $C=\{x_i\}$ and a clause $C'=\{\overline{x_i}\}$. Modify the randomized procedure above to improve the
approximation factor from $1/2$ to at least $0.6$. That is, change the algorithm so that the expected
number of clauses satisfied by the process is at least $0.6k$.

\begin{solutionbox}{\stretch 1} \vspace{1em} 

  % begin your solution here %

\end{solutionbox}

\newpage

\part Give a randomized polynomial-time algorithm for the general MAX SAT Problem, so that the
expected number of clauses satisfied by the algorithm is at least a $0.6$ fraction of the maximum
possible. (Note that, by the example in part (a), there are instances where one cannot satisfy more
than $k/2$ clauses; the point here is that we'd still like an efficient algorithm that, in expectation,
can satisfy a $0.6$ fraction of the maximum that can be satisfied by an optimal assignment.)

\begin{solutionbox}{\stretch 1} \vspace{1em} 

  % begin your solution here %
  
\end{solutionbox}

\end{parts}
\newpage


  \question \textit{Kleinberg, Jon. Algorithm Design (p. 789, q. 10).} 
  
  Consider a very simple online auction system that works as follows. There are $n$ \textit{bidding agents}; agent $i$
has a bid $b_i$, which is a positive natural number. We will assume that all bids $b_i$ are distinct from one
another. The bidding agents appear in an order chosen uniformly at random, each proposes its bid $b_i$ in
turn, and at all times the system maintains a variable $b^*$ equal to the highest bid seen so far. (Initially
$b^*$ is set to $0$.) What is the expected number of times that $b^*$ is updated when this process is executed,
as a function of the parameters in the problem?



  \begin{solutionbox}{\stretch 1} \vspace{1em} 

    % begin your solution here %
  
  \end{solutionbox}

\newpage

\question Recall that in an undirected and unweighted graph $G = (V,E)$, a cut is a partition of the vertices
$(S, V \backslash S)$ (where $S \subseteq V$ ). The size of a cut is the number of edges which cross the cut (the number of
edges $(u, v)$ such that $u \in S$ and $v \in V \backslash S$). In the MAXCUT problem, we try to find the cut which has
the largest value. (The decision version of MAXCUT is NP-complete, but we will not prove that here.)
Give a randomized algorithm to find a cut which, in expectation, has value at least $1/2$ of the maximum
value cut.

  \begin{solutionbox}{\stretch 1} \vspace{1em} 

     % begin your solution here %

  \end{solutionbox}

\end{questions}

\end{document}




